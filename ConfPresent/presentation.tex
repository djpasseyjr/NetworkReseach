\documentclass[40pt]{article}
\usepackage{cmbright}
\usepackage{amsmath}
\setlength{\parindent}{0pt}
\usepackage[left=2cm, right=4cm, top=2cm]{geometry} 

\begin{document}
\huge
(Bunimocich, Smith, Webb 2018)

Let $G=(V,E,\omega)$ and let $C_1$, $C_2$...$C_m$ be the strongly connected components in the graph $G$ with $B$ removed. Then,
\[
\sigma(S_B(G)) = \sigma(G)\, \cup\, \sigma(S_1)^{n_1 - 1} \, \cup \, \sigma(S_2)^{n_2 - 1} \, \cup ... \cup \,  \sigma(S_k)^{n_k - 1}
\]
where each $n_i$ is the number of branches that contain $S_i$.

\vspace{1cm}

(Bunimocich, Passey, Smith, Webb 2018)

1. The centrality of nodes in the base set is preserved

\vspace{.5cm}


2. If $Z_1$ and $Z_2$ are copies of $Z$ in $S_B(G)$ then the centrality of corresponding nodes in $Z_1$ and $Z_2$ are the same if $Z_1$ and $Z_2$ have the same incoming branch.

\vspace{.5cm}

3. If the set \{$Z_1$, $Z_2$, ... $Z_k$\} is the set of all copies of $Z$ in $Sp(G)$ that have the same outgoing branch, then their centralities sum to the centrality of $Z$ in the original graph.


\end{document}
$
\begin{bmatrix}
1 \\ .47
\end{bmatrix}
\begin{bmatrix}
1.6 \\ 1.6
\end{bmatrix}
$

$
\begin{bmatrix}
4.9 \\ 4.4 \\ 4.4
\end{bmatrix}
\begin{bmatrix}
3.8 \\ 1.8 \\ 1.8
\end{bmatrix}
\begin{bmatrix}
3.3 \\ 3 \\ 3
\end{bmatrix}
\begin{bmatrix}
2.6 \\ 1.2 \\ 1.2
\end{bmatrix}
\begin{bmatrix}
1.6 \\ 1.4 \\ 1.4
\end{bmatrix}
\begin{bmatrix}
1.2 \\ .6 \\ .6
\end{bmatrix}
$

The spectrum of $G$, or $\sigma(G)$, is the eigenvalues of the adjacency matrix associated with G.
\vspace{1cm}

Let $Z$ be a strongly connected component of $G$ with the base removed. 

\vspace{.5cm}
1. If $Z_1$ and $Z_2$ are copies of $Z$ in $Sp(G)$ then the centrality of corresponding nodes in $Z_1$ and $Z_2$ are the same if $Z_1$ and $Z_2$ have the same incoming branch.

\vspace{.5cm}

2. If the set \{$Z_1$, $Z_2$, ... $Z_k$\} is the set of all copies of $Z$ in $Sp(G)$ that have the same outgoing branch, then their centralities sum to the centrality of $Z$ in the original graph.

\huge
$\sigma(G)$ = \{ 
   $1.4,
  -1,
  -.23+.79i,
  -.23-.79i$\}
\vspace{1cm}

$\sigma(Sp(G))$ = \{$
   1.4,
  -1,
  -.23+.79i,
  -.23-.79i$\} $\, \cup \, $\{$
   1,
  -1$ \}

\vspace{1cm}

\Huge
$G = $

\vspace{1cm}
$Sp(G) = $


\vspace{1cm}

Let $G = (V,E,\omega )$ with structural set $B \subset V$. Assume that $S_1$, $S_2$...$S_k$ are the strongly connected components of $G$ with $B$ removed. Then the eigenvalues of the specialized graph,
\begin{center}
$\sigma(Sp(G)) = $
\\
\vspace{.5cm}
$\sigma(G) \cup \sigma(S_1)^{n_1 - 1} \cup \sigma(S_2)^{n_2 - 1} \cup ... \cup  \sigma(S_k)^{n_k - 1}$
\end{center}


where $n_i$ is the number of component  branches that contain $S_i$.


Let $G = (V,E,\omega)$ the spectrum of $G$, $\sigma(G)$ is the eigenvalues of the adjacency matrix associated with G.
\vspace{1cm}



$V = \{1,2,3\}
\vspace{1cm}
\\
E = \{(1,2),(1,3),(2,3),(3,2)\}
\vspace{1cm}
\\
\omega((1,2)) = \omega((3,2)) = 1
\vspace{1cm}
\\
\omega((1,3)) = \omega((2,3)) = 5$

A graph $G$ is a triple $G = (V,E,\omega)$ where $V$ is a set of vertices, $E$ is a set of edges and $\omega $ is a function that associates each edge with a weight.
\vspace{1cm}

\textsf{
The adjacency matrix for a network is a matrix $ A$ where $ A_{ij}$ equals the weight of the link from node $j$ to node $i$.
}

\[
A = 
\begin{bmatrix}
0 & 0 & 0 \\
1 & 0 & 1 \\
5 & 5 & 0 
\end{bmatrix}
\]

\begin{bf}Definition (Strongly Connected)
\end{bf}
\\
Let $\mathsf G = (V,E,\omega)$. Let $C \subset V$. If for every $v_i,v_j \in C$ there exists a path from $v_i$ to $v_j$, then we call the graph $S = G|_C$ a strongly connected component of $G$.

$V = \{1,2,3\}
\\
E = \{(1,2),(1,3),(2,3),(3,2)\}
\\
\omega(e) = 1\quad \text{For all edges}$



$A \,\, \text{is an adjacency matrix} \\
\text{for} \,G \,\text{if:}\\
A_{ij} = 1 \\
\text{when there is an edge }\\
\text{from vertex}\, j\, \text{to vertex} \, i$


$
A = \begin{bmatrix}
0 & 0 & 0 \\
1 & 0 & 1 \\
1 & 1 & 0 \\
\end{bmatrix}
$