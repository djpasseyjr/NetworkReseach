\documentclass{paper}
\usepackage{amsmath}
\usepackage{parskip}
\begin{document}

\begin{table*}[]
\centering
\label{table:3}
\begin{tabular}{|| c || c | c ||}
\hline
Method &  Mean & Standard Deviation \\
\hline
\textbf{Original} & \textbf{15883.35} &  \textbf{21995.36} \\
\textbf{Random Specialization} & \textbf{38.89} & \textbf{187.57} \\
Random Specialization Control &  825.82 & 3022.73 \\
\textbf{Impact Specialization} &  \textbf{19.91} & \textbf{67.21}  \\
Impact Specialization Control &  33.73 & 183.31 \\
\hline
\end{tabular}

\end{table*}

Original: Reservoir computer was initialized with a random graph of 30 nodes and a .12 probability of edges. The average error and standard deviation are so large because this model produced many networks did a terrible job learning the problem. The specialized networks did not produce any networks that preformed this badly.

Random Specialization: Three random nodes from the original graph were specialized and the resulting network was used in the reservoir computer.

Random Specialization Control: Each reservoir computer was initialized with a random graph containing the same number of nodes and edges as the specialized graphs

Impact Specialization: The three nodes whose activations were the most useful in learning were specialized.

Impact Specialization Control: Each reservoir computer was initialized with a random graph containing the same number of nodes and edges as  the impact specialized graphs
\end{document}