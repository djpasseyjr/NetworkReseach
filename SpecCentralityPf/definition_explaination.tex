\documentclass{paper}
\usepackage{amsmath}
\usepackage{amsthm}
\usepackage{parskip}
\begin{document}

\title{Definition Issues}
\maketitle

The definitions in your paper are definitely more elegant than mine, and make the theorems a lot nicer. However, the issue I'm running into is the one I've been having a difficult time with for months. It is this:

The main result I need from my lemma is that 
\[ X_{ij} = \sum_{\alpha \in In(S_j,S_i)} P(\alpha ). \]
In other words, the sub diagonal entries of $\rho I-G)^{-1}$ are the sum of centrality transfer matrices from strongly connected component $S_j$ to $S_i$. 


This is a way of explaining how centrality moves from component to component and in my main result, I will work backward through the components to the base and connect invariant base centralities to the centralities in the components.


A subtle detail is that later, I need alpha to be a branch from the \begin{bf}base\end{bf} to a strongly connected component, so $In(.\, , \, .)$ needs to work from base to s.c. component as well as s.c. component to s.c. component.


To do this I need a definition of an incoming branch that works for incoming branches between strongly connected components and incoming branches from the base to a strongly connected component.


It's really tricky and I haven't figured out how to do it perfectly. The way I was thinking of doing it was by using a really general definition of a component branch that could begin at a subgraph and end at a component (or at the same subgraph). Then I could define centrality transfer matrixes for both base to component (because the base is a subgraph) and component to component.

I liked your way of connecting the definition to component branches, and I thought that maybe we could make a definition for \begin{it}component subbranches \end{it}, that could be any sub sequence of a component branch that begins and ends with a subgraph. 

The problem with that idea is that my lemma is not connected to specialization, and doesn't include the notion of a base, so component branches don't seem to fit in. I could also drop centrality transfer matrixes from the proof, but I'm worried that it would get a lot messier than it already is.

I included my definitions below so that you can see what I tried to do. 


\subsubsection*{Definition (Incoming branch)}

Given a graph $G = (V,E,\omega)$, a set $B \subset V$ and a strongly connected component $Z$ of $G|_{\overline{B}}$, let $S_1,S_2 \cdots S_k$ be strongly connected components of $G|_{\overline{B}}$. Let $H$ represent the subgraph $G|_B$.

If there exist edges $e_0 \cdots e_m$  such that,

(i) $e_0$ is an edge from $H$ to $S_1$

(ii) $e_j$ is an edge from $S_j$ to $S_{j+1}$ for $1 \leq j \leq k-1$

(iii) $e_k$ is an edge from $S_k$ to $Z$

The we call the ordered set $\alpha$ = \{$H$, $e_0$, $S_1$, $e_1$, $S_2,...,S_k$, $e_k$, $Z$\} an incoming branch from $B$ to $Z$. We let $In(H,Z)$ denote the set of all incoming branches from $H$ to $Z$ in $G$.


\subsection*{Definition (Subgraph generated by an ingoing branch)}
Let $G = (V,E,\omega)$ with $B \subset V$ and $Z$ a strongly connected component of $G|_{\overline{B}}$. If $\alpha = $\{$B$, $e_0$, $S_1$, $e_1$, $S_2,...,S_k$, $e_k$, $Z$\}$ \in In(B,Z)$, then the subgraph generated by $alpha$ is the graph consisting of all all strongly connected components and edges in $\alpha$ together with $G|_{B}$

\subsection*{Definition (Centrality Transfer Matrix)}
Let $G = (V,E,\omega)$ have a centrality vector and associated spectral radius $\rho$, let $B \subset V$ and let $Z$ be a strongly connected component of $G|_B$. If $\alpha$ = \{$B$, $e_0$, $S_1$, $e1$, $S_2$,...,$S_k$, $e_k$, $Z$ \} is an incoming path from B to Z then

\[
P(\alpha) = (\rho I - Z)^{-1}Y_k(\rho I - S_k)^{-1}Y_{k-1}\cdots (\rho I - S_1)^{-1}Y_0(\rho I - B)^{-1}
\]
is a centrality transfer matrix of $\alpha$, if
\[
 \begin{bmatrix}
B \\
Y_0 & S_1 \\
    & Y_1 & S_2 \\
    &     &     & \ddots \\
&&& Y_{k-1} & S_k \\
&&&& Y_k & Z 
\end{bmatrix}
\]
is an adjacency matrix for the subgraph generated by $\alpha$.

\subsection*{Lemma}
Assume G = $(V,E,\omega)$ is not strongly connected. Let If $\rho > \max\{|\lambda| : \lambda \in \sigma(G)\}$ then,
\[
(\rho I - A)^{-1} = 
\begin{bmatrix}
(\rho I - S_1)^{-1} \\
X_{21} & (\rho I - S_2)^{-1} \\
X_{31} & X_{32} & (\rho I - S_3)^{-1} \\
\vdots & \vdots & \ddots & \ddots \\
X_{k1} & X_{k2} & & X_{kk-1} & (\rho I - S_k)^{-1} \\
\end{bmatrix}
\] 

where $A$ is an adjacency matrix of $G$, 
$S_1$, $S_2$,...,$S_k$ are all adjacency matrices of the strongly connected components of $G$, and each 
\[ X_{ij} = \sum_{\alpha \in In(S_j,S_i)} P(\alpha ). \]

\end{document}