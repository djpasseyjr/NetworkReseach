\documentclass{paper}
\usepackage{amsmath}
\usepackage{amsthm}
\usepackage{parskip}
\begin{document}

\section*{Centrality and Specialization}

I read through the changes you made and they seem great. 

Dallas does use $A_{ij}$ is an edge from $v_i$ to $v_j$ so I'll need to change my proofs. 

I looked at the section on dynamic networks and it looks like you were using the $A_{ij}$ is an edge from $v_j$ to $v_i$, specifically after definition 4.2 in the equation:

\begin{equation}\label{eq:netclass}
F_i(\mathbf{x})=\sum_{j=1}^n A_{ij}f_{ij}(x_j), \quad \text{for} \quad i\in N=\{1,2,\dots,n\}
\end{equation}

To me, it looks like you are using $A_{ij}$ to weight the effect $x_j$ has on $x_i$. Thus it seems to represent the edge weight from $v_j$ to $v_i$.

This is important because I'm concerned that I need to define eigenvectors for the transpose of the adjacency matrix. When we define the adjacency  matrix $A$ the way Dallas did, the dynamics should technically be $F(\mathbf{x})= A^{T} \mathbf{x}$ (for example), then the eigenvectors that really matter are the eigenvectors of $A^{T}$ and my results should be about those. (Hope that made sense...)


Also, I think that we do get a little more than just centrality for free, so why don't we go ahead an include a lambda in the eigenvector transfer matrix to make it more general. Maybe $T_\lambda(\beta,Z)$. I'll think more carefully about this as well as the result for weighted graphs as I rewrite my proof.

I've put together a plan for using the updated definitions to prove my results in the appendix. Let me know what you think.


\subsection*{Theorem}\textbf{(Eigenvectors of Specialized Graphs)}
Let $G=(V,E,\omega)$ be a graph and $B\subseteq V$ a base.\\
(i) If $(\lambda,\mathbf{x})$ is an eigenpair of the graph $G$ and $\lambda\notin\sigma(G|_{\bar{B}})$ then there is an eigenpair $(\lambda,\mathbf{y})$ of $\mathcal{S}_B(G)$ such that $\mathbf{x}_B=\mathbf{y}_B$.\\
Furthermore, suppose $G$ is strongly connected with positive edge weights and let $\mathbf{u}$ be a leading eigenvector of $G$. Also, let $Z$ be a strongly connected component of $\beta\in\mathcal{B}_S(G)$. Then there is a leading eigenvector $\mathbf{v}$ of $\mathcal{S}_B(G)$ such that the following hold.\\
(ii) For all $Z_i\in\mathcal{C}(Z)$ the eigenvector restriction 
\[
\mathbf{v}_{Z_i}=T(\beta,Z)\mathbf{v}_B.
\] 
Hence, if $Z_i,Z_j\in\mathcal{C}(Z)$ have the same incoming branch then $\mathbf{v}_{Z_i}=\mathbf{v}_{Z_j}$.\\
(iii) For $Z_i\in\mathcal{C}(Z)$ let $\cup_{k=1}^\ell\{Z_k\}$ be the copies of $Z$ that have the same outgoing branch as $Z_i$. Then
\[
\mathbf{u}_Z=\sum_{k=1}^{\ell}\mathbf{v}_{Z_k}=\sum_{k=1}^{\ell}T(\beta,Z)\mathbf{v}_B.
\]

\begin{proof}

(ii) with the new definition this will be really easy

(iii) I'll need the lemma below. I've reworked it to fit with the new definitions
\end{proof}

\subsection*{Definition} Let $S_i,S_j$ be strongly connected components of a graph $G$. Let $S$ be the set of vertices containing in $S_i$ and $S_j$. We let
\[
B_{in}(S_i\,,S_j) = \{ In(\beta,S_j)\, |\, \beta \in \mathcal{B}_{S_i}(G) \}
\]
\subsection*{Lemma}Assume $G = (V,E,\omega)$ is not strongly connected and let $\lambda \notin \sigma(G)$. If $S_1$, $S_2$,...,$S_k$ are the strongly connected components of G, then there exists an adjacency matrix $A$ for $G$ such that 
\[
(\rho I - A)^{-1} =
\begin{bmatrix}
X_{11} & X_{12} & \cdots & X_{1k} \\
 & X_{22} & \cdots & X_{2k} \\
	& & \ddots & \vdots \\
	X_{kk} & & & 
\end{bmatrix}
\]
Where
\[ X_{ij} = \sum_{\beta \in B_{in}(S_j,S_i)} T(\beta,S_i)(\rho I - S_j)^{-1} 
\]
if $i < j$ and
\[ X_{ij} = (\rho I - S_j)^{-1} 
\]
if $i=j$

\end{document}
